% -------------------------------------------------------
% In dieser Datei sollten eigentlich keine Veränderungen mehr
% notwendig sein.
% -------------------------------------------------------

\thispagestyle{empty}

% -------------------------------------------------------
\newcommand{\srhfakultaet}{Fakultät für Information, Medien und Design}%
\newcommand{\srhstudiengang}{Big Data und Business Analytics}%
\newcommand{\srhtyp}{Projektbericht}
\newcommand{\srhmaster}{Master of Science (M.Sc.)}
\newcommand{\srhkoerperschaft}{SRH Heidelberg}
\newcommand{\srhautorbib}{\srhautornname, \srhautorvname} % Autor Nachname, Vorname
\newcommand{\srhautor}{\srhautorvname \ \srhautornname} % Autor Vorname Nachname
\newcommand{\srhautorzwei}{\srhautorzweivname \ \srhautorzweinname} % Autor Vorname Nachname
\newcommand{\srhherbert}{Prof. Dr. Herbert Schuster} % Autor Vorname Nachname
\newcommand{\srhfrank}{Frank Schulz} % Autor Vorname Nachname
\newcommand{\srhbarbara}{Prof. Dr. Barbara Sprick}
\newcommand{\srhbianca}{Bianca Staffen}

% Daten in die Standard-Felder von KOMA-Script eintragen
\titlehead{\srhtyp\ in\  \srhstudiengang}
\subject{}
\title{\srhtitel}
\author{\srhauthor}
\date{\small{\srhdatum}}

% Daten für das fertige PDF-Dokument
\hypersetup{
  pdftitle={\srhtitel},  % Titel des Dokuments
  pdfauthor={\srhautor},              % Autor
  pdfsubject={\srhtyp\ in\ \srhstudiengang},                % Thema
  pdfkeywords={\srhtitel}         % Schlüsselworte
}

\newlength{\bindekorrektur}
\newlength{\seitenanfang}
\newlength{\seitenbreite}

\setlength{\bindekorrektur}{-46mm}   % Korrektur der horizontalen Position
\setlength{\seitenanfang}{0mm}       % Korrektur der vertikalen Position
\setlength{\seitenbreite}{297mm}

\begin{figure}[h]
  \flushright
  \includegraphics[width=7cm]{srh_logo.png}
\end{figure}


% Titel der Arbeit
\begin{textblock*}{\seitenbreite}(\bindekorrektur,\seitenanfang + 62mm) % 4,5cm vom linken Rand und 6,0cm vom oberen Rand
  \centering\Large\sffamily
  \vspace{4mm} % Kleiner zusätzlicher Abstand oben für bessere Optik
  \textbf{\srhtitel}
\end{textblock*}%

% Projektbericht
\begin{textblock*}{\seitenbreite}(\bindekorrektur,\seitenanfang + 103mm)
  \centering\large\sffamily
  \srhtyp
  \vspace{2mm} \\
  von
\end{textblock*}

% Name
\begin{textblock*}{\seitenbreite}(\bindekorrektur,\seitenanfang + 130mm)
  \centering\large
  \textbf{\srhautor} \\
  \vspace{2mm}
  Matrikelnummer: \srhmatnr \\
  \vspace{5mm}
  und
\end{textblock*}

% Name 2
\begin{textblock*}{\seitenbreite}(\bindekorrektur,\seitenanfang + 155mm)
  \centering\large
  \textbf{\srhautorzwei} \\
  \vspace{2mm}
  Matrikelnummer: \srhmatnrzwei
\end{textblock*}

% Datum
\begin{textblock*}{\seitenbreite}(\bindekorrektur,\seitenanfang + 185mm)
  \centering\large
  \textsf{\srhdatum}
\end{textblock*}

% Fakultät
\begin{textblock*}{\seitenbreite}(\bindekorrektur,\seitenanfang + 205mm)
  \centering\large\sffamily
  \srhkoerperschaft \\
  \vspace{2mm}
  \srhfakultaet \\
  \vspace{2mm}
  \srhstudiengang
\end{textblock*}

% Dozent(en)
\begin{textblock*}{\seitenbreite}(\bindekorrektur,\seitenanfang + 250mm)
  \centering\large\sffamily
  Dozent \\
  \vspace{2mm}
  \srhfrank
\end{textblock*}

% Bibliographische Informationen
\null\newpage
\thispagestyle{empty}

\newcommand{\srhbib}{\begin{small}\textbf{\srhautorbib}: \\ \srhtitel \ / \srhautor. \ -- \\ \srhtyp, \srhort \: \srhkoerperschaft, \srhjahr. \pageref{lastpage} Seiten.\end{small}}
