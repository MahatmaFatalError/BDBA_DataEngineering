\section{\acl{ITSM}}
Ein Service stellt jegliche Art von Dienstleistung eines Unternehmens dar, welche dem Konsumenten einen Mehrwert bietet.
\footnote{\cite[Def. von Service][S.10]{Farenden2012}}
\newline
Bei der Nutzung eines Services, zum Beispiel bei einer Anfrage bei einer Service Hotline, erwartet der Nutzer des Services, dass der Ansprechpartner das eigene Problem versteht, analysiert und eine
sofortige Lösung des Problems oder eine eventuelle Alternative bereitstellt.
Für Unternehmen bedeutet dies, Eingehen auf Kundenwünsche und die regelmäßige Anpassung des Services.
Ein Unternehmen sollte in jedem Fall das Ziel verfolgen den Kunden zu verstehen um ihm eine zufriedenstellende Lösung bereitzustellen, die ihm einen Mehrwert bietet.
Um dies zu ermöglichen ist es von großer Bedeutung, dass im Hintergrund ablaufenden Prozesse bestmöglich funktionieren und den Anwender bei seiner täglichen Arbeit unterstützen.

\index{\acl{ITSM}}
Das \ac{ITSM} beschreibt Methoden und Maßnahmen, die es einem IT-Dienstleister erleichtern Prozesse für einen IT Service zu planen, zu implementieren, zu überwachen und zu steuern.
\section{\acl{ITSM} nach \acs{ITIL}}
Die \ac{ITIL} bietet Best Practices für verschiedene Bereiche des IT Service Managements an.
Es handelt sich dabei um eine Sammlung von sechs Büchern, die bisher erprobte und bewährte Erkenntnisse, Modelle und Architekturen enthalten.
Dabei ist zu bedenken, dass \ac{ITIL} keine offizielle Norm darstellt, die ein Unternehmen einführen muss um erfolgreich \ac{ITSM} einzuführen.
Die Bücher von \ac{ITIL} stellen lediglich Best Practices und Prozesse zusammen die als Hilfestellungen zur Gestaltung von Geschäftsprozessen zu verstehen sind.
Dies erkennt man auch daran, dass die \ac{ITIL} nur vorschlägt \textbf{Was} zu tun ist um solche Prozesse erfolgreich umzusetzen.
\textbf{Wie} die Prozess dann schlussendlich umzusetzen sind das bleibt die Aufgabe des jeweiligen Unternehmens und ist abhängig von der den Anforderungen und den Bedürfnissen des Unternehmens bzw. des Kunden.

Das Herzstück von \ac{ITIL} bilden insgesamt fünf Kernbereiche:
\begin{itemize}
	\item Service Strategy
	\item Service Design
	\item Service Transition
	\item Service Operation
	\item Continual Service Improvement
\end{itemize}
Diese orientieren sich an den einzelnen Phasen des Service Lebenszyklus`. Der Ursprung aller Serviceprozesse ist die \textit{Service Strategy}. Erst wenn diese Phase beendet wird kann mit den restlichen Phasen begonnen werden.
Sobald die Strategie des Services festgelegt ist wird in der Phase \textit{Service Design} das Design der Strategie entwickelt und anschließend, in der \textit{Service Transition}, das endgültige Produkt erstellt und in den Live-Betrieb überführt.
Die Phase \textit{Service Operations} ist für das Management des IT Services im Live-Betrieb zuständig.
Die kontinuierliche Verbesserung umschließt kreisförmig alle Phasen und verdeutlicht somit die dauerhaften Einfluss der Phase \textit{Continual Service Improvement}, auf welche man in allen Phasen Bezug nehmen sollte.
\fullref{fig:slc} stellt diesen Prozess dar.

\bild{../bilder/grundlagen/itsm/ServiceLifecycle.jpg}{Service Lifecycle}{slc}{\cite[vgl.][S.37ff]{Farenden2012}}{0.5}


In den nachfolgenden Kapiteln werden die einzelnen Phasen des Service Lifecycles näher erläutert um ein Grundverständnis für das \ac{ITSM} zu schaffen.
\newpage
\subsection{Service Strategy}
\footnotetext{\cite[vgl.][S.17]{Farenden2012}}
Bevor ein neuer Service entworfen, eingeführt und als Dienstleistung angeboten werden kann ist es unerlässlich, diesen Service zunächst zu planen und auf Machbarkeit hin zu untersuchen.
Der Bereich der Servicestrategie konzentriert sich auf diese erste Phase der Erstellung eines Services und spricht Empfehlungen zu Service Management Richtlinien, Strukturen und Verfahren und Prozesse, die für den kompletten Service Lebenszyklus gültig sind, aus.
Das Ziel der Servicestrategie ist die Erwartungen des Kunden durch den neuen Service optimal zu bedienen, neue Geschäftsmöglichkeiten zu analysieren und Kosten und Risiken zu bewerten.
Des Weiteren ist es die Aufgabe der Servicestrategie die Marktposition des Unternehmens, durch den neuen Service, zu halten und zu stärken.
\subsubsection{Teilprozesse der Service Strategy}
\paragraph{Strategy Management für IT Services}
Der Prozess Strategy Management für IT Services strebt die Entwicklung einer Strategie auf Basis der Unternehmensstrategie an.
Letztendlich soll der Prozess eine gemeinsame Grundlage schaffen um das Bestehen des Unternehmens auf lange Hinsicht zu sichern.
