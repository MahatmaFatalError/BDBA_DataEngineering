\subsection{Data Ingestion via CSV Datei}
Als offline-fähige Alternative zur \ac{SODA} API gibt es einen weiteren Kafka Producer in Java. Dieser liest die Nachrichten zeilenweise aus einer lokalen \ac{CSV} Datei ein und publiziert die Datensätze als \ac{JSON} einzeln an einen Kafka Topic. Die \ac{CSV} Datei wurde vorher aus dem NYC OpenData Portal runtergeladen.

Der Programmablauf lässt sich in wenigen Stichpunkten beschreiben:
\begin{enumerate}
	\item Verbindung des Kafka Producers zum Kafka Cluster konfigurieren. Dazu zählen hauptsächlich Host und Port sowie die Datentypen, um Schlüssel und Wert der Nachrichten zu serialisieren.  
	\item Zeilenweises einlesen der \ac{CSV} Datei und dabei jeweils den Datensatz in einen \ac{JSON}-String transformieren, diesen \code{String} als Wert in einen \code{ProducerRecord} setzen und an den bestimmten Kafka Topic senden. Um einen realen Stream zu simulieren wartet der Producer-Thread pro verarbeiteten Datensatz eine zufällige Wartezeit zwischen 0 bis 2 Sekunden.
\end{enumerate}

%\begin{tcblisting}{width=19cm,listing only,blank,tikz={rotate=90},listing options={basicstyle=\ttfamily}}
\lstinputlisting[language=Java, firstline=35, lastline=67, label=listing:javacsvproducer,
	captionpos=b,
	caption=Auszug aus \code{com.srh.bdba.dataengineering.MyProducer}]{../java/src/main/java/com/srh/bdba/dataengineering/MyProducer.java}
%\end{tcblisting}


