\subsection{Data Retrieval mit Jupyter}\label{subsec:jupyter}
Das Jupyter Notebook bietet die Möglichkeit mit \textit{Magic-Commands} direkt aus dem Notebook heraus
eine Verbindung mit der Datenbank aufzubauen und \ac{SQL} Statements abzusetzen.

Einfache Tabellen werden direkt in Jupyter visualisiert wohingegen komplexere Visualisierungen wie \zb{}
Balkendiagramme oder Geo Plots mit Python Bibliotheken dargestellt werden müssen.
\\
Für die Darstellung in Jupyter werden dann sog. \textit{Widgets} installiert und aktiviert.

Folgende Python Bibliotheken wurden installiert um \ac{SQL} Statements in Jupyter ausführen und visualisieren zu können.
\begin{itemize}
  \item \code{ipython-sql}
  \item \code{sqlalchemy} (wird von ipython-sql benötigt)
  \item \code{bokeh}
  \item \code{gmaps}
\end{itemize}

Mit \code{ipython-sql} werden \ac{SQL} \textit{Magic-Command} in Jupyter aktiviert.

\code{ipython-sql} nutzt \code{sqlalchemy} um sich mit der Datenbank zu verbinden.
Ein abgesetztes \ac{SQL} Statement lässt sich entweder direkt in Jupyter ausgeben
oder einer beliebigen Variable zuordnen die dann weiterverarbeitet werden kann.

\fullref{fig:jupyterEins} zeigt ein \ac{SQL} Statemement mit \textit{Magic-Command}
Das Resultat wird direkt in einem Jupyter Notebook als Tabelle gerendert.

\fullref{fig:jupyterZwei} zeigt die Zuweisung eines Resultates zur Variable \code{works}.

\bild{jupyter_1.png}{direkte \ac{SQL} Ausgabe in Jupyter}{jupyterEins}{\autocite{MagicExample}}{0.75}

\bild{jupyter_2.png}{Zuweisung der \ac{SQL} Ausgabe einer Variable}{jupyterZwei}{\autocite{MagicExample}}{0.75}

 \code{bokeh} ist eine Python Bibliothek um viele Arten der Visualisierung umzusetzen
 wie \zb{} Balkendiagramme, Scatter Plots, Geo Maps oder Zeitreihen.

 Die Visualisierung von Geo Daten erfolgt mit der Bibliothek \code{gmaps}.\\
 Diese greift auf die Karten von Google Maps zu erlaubt es die Geo Daten in einem Layer über einen
 beliebigen Kartenausschnitt zu legen.\\
 Der Vorteil von \code{gmaps} gegenüber \code{bokeh} ist
 zum einen die Nutzung des Kartenmaterials von Google aber auch die interaktive Nutzung
 des Kartenauschnitts mit \zb{} Google StreetView.

Folgende Fragestellungen wurden im Rahmen von Data Retrieval beantwortet.

\begin{enumerate}
  \item Zeige alle Beschwerdetypen die häufiger als 400 aber seltener als 8000 Mal gemeldet wurden?
  \item Wie lautet die Beschreibung der häufig vorkommenden Service Requests?
  \item An welchen Orten von New York City wurden Service Request vom Typ 'Noise - Residential' abgesetzt?
  \item Wieviele Service Requests sind im Jahr 2017 eingegangen? Gruppiert nach Tag und Sortiert nach dem Erstellungsdatum.
  \item Wie lange wurde eine Art von Service Request im Durschnitt bearbeitet? Wie lange war die minimale und maximale Bearbeitungszeit?
\end{enumerate}

Nachfolgender Abschnitt listet die \ac{SQL} Statements zu jeder Fragestellung sowie ein dazugehöriges Beispiel.

\textbf{zu 1.}
\newline
\sql{SELECT complaint\_type, COUNT(complaint\_type) FROM service\_request GROUP BY complaint\_type HAVING COUNT(complaint\_type) > 400 AND COUNT(complaint\_type) < 8000}

\bildhochkant{select_1.png}{Tabelle mit allen Beschwerdetypen}{srt}{eigene Darstellung}

\textbf{zu 2.}
\newline
\sql{SELECT descriptor, COUNT(descriptor) FROM service\_request  WHERE descriptor IS NOT NULL GROUP BY descriptor ORDER BY count DESC}

\bild{select_2.png}{Auszug aus den meisten Service Requests}{toprequests}{eigene Darstellung}{0.75}

\textbf{zu 3.}
\newline
\sql{SELECT longitude, latitude FROM service\_request WHERE complaint\_type = 'Noise - Residential' and latitude IS NOT NULL and longitude IS NOT NULL}

\bild{select_3.png}{Heatmap der Lärmquellen in New York}{rnnyc}{eigene Darstellung}{0.75}

\textbf{zu 4.}
\newline
\sql{SELECT date\_trunc('day', created\_date) AS dd, COUNT(created\_date) as daily\_sum FROM service\_request where EXTRACT(year from created\_date) = '2017' GROUP BY dd ORDER BY date\_trunc('day', created\_date)}

\bild{select_4.png}{Timeline aller Service Requests in 2017}{sr2017}{eigene Darstellung}{0.75}

\textbf{zu 5.}
\newline
\sql{SELECT AVG(closed\_date - created\_date) AS avg\_duration, MIN(closed\_date - created\_date) AS min\_duration, MAX(closed\_date - created\_date) AS max\_duration, complaint\_type FROM service\_request
where created\_date IS NOT NULL and closed\_date IS NOT NULL
GROUP BY complaint\_type
HAVING MAX(closed\_date - created\_date) < INTERVAL '365 days' and MIN(closed\_date - created\_date) > '00:00:00'
ORDER BY avg\_duration asc}

\bild{select_5.png}{Lageparameter zur Bearbeitungsdauer von Service Requests}{lparam}{eigene Darstellung}{0.75}
