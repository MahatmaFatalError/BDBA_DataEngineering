\subsection{Data Retrieval mit Jupyter}
Das Jupyter Notebook bietet die Möglichkeit mit \textit{Magic-Commands} direkt aus dem Notebook heraus
eine Verbindung mit der Datenbank aufzubauen und \ac{SQL} Statements abzusetzen.
Einfache Tabellen werden direkt in Jupyter visualisiert wohingegen komplexere Visualisierungen wie \zb{}
Balkendiagramme oder Geo Plots mit Python Bibliotheken dargestellt werden müssen.
Für die Darstellung in Jupyter werden sog. \textit{Widgets} installiert und aktiviert.


Folgende Python Bibliotheken wurden installiert um \ac{SQL} Statements in Jupyter ausführen und visualisieren zu können.
\begin{itemize}
  \item ipython-sql
  \item sqlalchemy (wird von ipython-sql benötigt)
  \item bokeh
  \item gmaps
\end{itemize}

Mit \code{ipython-sql} werden \ac{SQL} \textit{Magic-Command} in Jupyter aktiviert.
\code{ipython-sql} nutzt \code{sqlalchemy} um sich mit der Datenbank zu verbinden.
Ein abgesetztes \ac{SQL} Statement lässt sich entweder direkt in Jupyter ausgeben
oder einer beliebigen Variable zuordnen die dann weiterverarbeitet werden kann.


 \code{bokeh} ist eine mächtige Python Bibliothek um viele Arten der Visualisierung umzusetzen
 wie \zb{} Balkendiagramme, Scatter Plots, Geo Maps oder Zeitreihen.


 Die Visualisierung von Geo Daten erfolgt mit der Bibliothek \code{gmaps}.
 Diese greift auf die Karten von Google Maps zu erlaubt es die Geo Daten in einem Layer über einen
 beliebigen Kartenausschnitt zu legen.
 Der Vorteil von \code{gmaps} gegenüber \code{bokeh} ist
 zum einen die Nutzung des Kartenmaterials von Google aber auch die interaktive Nutzung
 des Kartenauschnitts mit \zb{} StreetView.


Folgende Fragestellungen wurden im Rahmen von Data Retrieval beantwortet.

\begin{enumerate}
  \item Zeige alle Beschwerdetypen die häufiger als 400 aber seltener als 8000 Mal gemeldet wurden?
  \item Wie lautet die Beschreibung der häufig vorkommenden Service Requests?
  \item An welchen Orten von New York City wurden Service Request vom Typ 'Noise - Residential' abgesetzt?
  \item Wieviele Service Requests sind im Jahr 2017 eingegangen? Gruppiert nach Tag und Sortiert nach dem Erstellungsdatum.
\end{enumerate}

Nachfolgender Abschnitt listet die \ac{SQL} Statements zu jeder Fragestellung sowie ein dazugehöriges Beispiel.


\textbf{zu 1.}
\newline
\sql{SELECT complaint\_type, COUNT(complaint\_type) FROM service\_request GROUP BY complaint\_type HAVING COUNT(complaint\_type) > 400 AND COUNT(complaint\_type) < 8000}

\bildhochkant{../bilder/select_1.png}{Beschwerdetypen - Tabelle}{srt}{eigene Darstellung}{0.5}

\textbf{zu 2.}
\newline
\sql{SELECT descriptor, COUNT(descriptor) FROM service_request  WHERE descriptor IS NOT NULL GROUP BY descriptor ORDER BY count DESC}

\bild{../bilder/select_2.png}{Top Requests - Bar Plot}{tsr}{eigene Darstellung}

\textbf{zu 3.}
\newline
\sql{SELECT longitude, latitude FROM service_request WHERE complaint_type = 'Noise - Residential' and latitude IS NOT NULL and longitude IS NOT NULL}

\bild{../bilder/select_3.png}{Residential Noise in NYC - Heatmap}{rnnyc}{eigene Darstellung}

\textbf{zu 4.}
\newline
\sql{SELECT date_trunc('day', created_date) AS dd, COUNT(created_date) as daily_sum FROM service_request where EXTRACT(year from created_date) = '2017' GROUP BY dd ORDER BY date_trunc('day', created_date)}

% \bild{../bilder/select_4.png}{Service Request 2017 - Timeline}{sr2017}{eigene Darstellung}
