\subsection{Data Retrieval mit Jupyter}
Das Jupyter Notebook bietet die Möglichkeit mit \textit{Magic-Commands} direkt aus dem Notebook heraus
eine Verbindung mit der Datenbank aufzubauen, \ac{SQL} Statements abzusetzen.
Einfache Tabellen werden direkt in dem Notebook visualisiert wohingegen Visualisierungen wie \zb
Balkendiagramme oder Geo Plots mit Python Bibliotheken dargestellt werden müssen.
Dafür müssen in Jupyter sog. \textit{Widgets} erstellt bzw. installiert werden.

Folgende Python Bibliotheken müssen noch installiert werden um die \ac{SQL} Statements in Jupyter ausführen und visualisieren zu können.
\begin{itemize}
  \item ipython-sql
  \item bokeh
  \item gmaps
\end{itemize}

Mit \textitbf{ipython-sql} werden das \%sql \textit{Magic-Command} in Jupyter aktiviert.
\textitbf{ipython-sql} nutzt \textitbf{sqlalchemy} um sich mit der Datenbaank zu verbinden.
Ein abgesetztes \ac{SQL} Statement lässt sich entweder direkt in Jupyter ausgeben
oder einer beliebigen Variable zuordnen die man dann weiterverarbeiten kann.

 \textitbf{bokeh} ist eine sehr mächtige Python Bibliothek um viele Arten der Visualisierung umzusetzen
 wie \zb Balkendiagramme, Scatter Plots, Geo Maps oder Zeitreihen.

 Die Visualisierung von Geo Daten erfolgt mit der Bibliothek \textitbf{gmaps}.
 Diese greift auf die Karten von Google Maps zu erlaubt es die Geo Daten in einem Layer über einen
 beliebigen Kartenausschnitt zu legen.
 Der Vorteil von \textitbf{gmaps} gegenüber \textitbf{bokeh} ist
 zum einen die Nutzung des Kartenmaterials von Google aber auch die interaktive Nutzung
 des Kartenauschnitts mit \zb StreetView


Im Rahmen von Data Retrieval wurden mehrere Fragestellungen über den Datensatz beantwortet.
Nachfolgend eine Auflistung aller Fragestellungen und den dazugehörigen \ac{SQL} Statements.

% TODO Bilder einfügen

\textbf{Wieviel Beschwerdetypen gibt es die häufiger als 400 aber seltener als 8000 Mal gemeldet wurden?}
\newline
\sql{SELECT complaint\_type, COUNT(complaint\_type) FROM service\_request GROUP BY complaint\_type HAVING COUNT(complaint\_type) > 400 AND COUNT(complaint\_type) < 8000}

\textbf{Wie lautet die Beschreibung der häufig vorkommenden Service Requests?}
\newline
\sql{SELECT descriptor, COUNT(descriptor) FROM service_request GROUP BY descriptor HAVING count(descriptor) > 8}

\textbf{Welche Orte von New York City wurde Service Request vom Typ 'Noise - Residential' abgesetzt?}
\newline
\sql{SELECT longitude, latitude FROM service_request WHERE complaint_type = 'Noise - Residential' and latitude IS NOT NULL and longitude IS NOT NULL}

\textbf{Wieviele Service Requests sind im Jahr 2017 aufetreten? Gruppirt nach Tag und sortiert nach dem Erstellungsdatum}
\newline
\sql{SELECT date_trunc('day', created_date) AS dd, COUNT(created_date) as daily_sum FROM service_request where EXTRACT(year from created_date) = '2017' GROUP BY dd ORDER BY date_trunc('day', created_date)}
