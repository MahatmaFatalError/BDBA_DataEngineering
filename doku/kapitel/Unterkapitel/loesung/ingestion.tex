\section{Data Ingestion}

Wie in \fullref{chap:einleitung} erläutert wird für die Erstellung des Prototyps NYC Open Data als Datenquelle genutzt.

NYC Open Data bietet verschiedene Datenätze an um an die unterschiedlichsten Informationen aus New York zu gelangen
wie \zb{} der Standort von öffentlichen Wi-Fi Hotspopts oder offene Stellenauschreibungen.\autocite{NYCOpenDataExample}

Innerhalb dieses Projektes wird der Datensatz mit dem Kürzel \textit{fhrw-4uyv} verwendet.
Er beinhält alle Service Request die seit 2011 von den Einwohner von New York City abgesetzt worden sind.

Beispielsweise kann mit diesem Datensatz herausgefunden werden wo welche Straßenlaternen in New York City ausgefallen sind
oder in welchem Haus zu welcher Uhrzeit gefeiert wurde da sich jemand über den Lärm beschwert hat.

In unserem Projekt beschaffen wir die Daten über eine \ac{CSV} Datei die man sich bei NYC Open Data herunterladen kann
und direkt über die \ac{SODA} \ac{API} die einen eigenen Endpunkt anbietet um die Service Requests abzufragen.

Die Data Ingestion per \ac{CSV} Datei setzte Julian Ruppel mit der Programmiersprache Java um und
den kontinuierlichen Datenstrom über die \ac{SODA} \ac{API} wurde von Johannes Weber mit der Programmiersprache Python
ausgelesen.

Beide Ansätze werden in diesem Kapitel beschrieben und anschließend miteinander verglichen.

\subsection{Data Ingestion via \acs{SODA} Schnittstelle}
Wie in \fullref{sec:arch} erläutert ist es die Aufgabe des Data Ingestion Prototyps zum einen Daten aus der externen Quelle auszulesen
aber auch die Daten direkt an die Apache Kafka Plattform weiterzuleiten und in ein Topic zu speichern.

Die Umsetzung des "Producers" erfolgte mit Python.
Zusätzlich wurden folgende Frameworks benutzt um die Implementierung des \textit{producers} zu unterstützen:

\begin{itemize}
  \item sodapy
  \item kafka-python
\end{itemize}

Sobald der \textit{producer} ausgeführt wird, wird der Nutzer gebeten ein Anfangs- und Enddatum anzugeben.
In diesem Zeitfenster werden dann alle Datensätze abgefragt und in das Topic "ServiceRequests" von Apache Kafka geschrieben.

\subsubsection{Quellcode}
Der \textit{producer} besteht insgesamt aus zwei Python Skripten:

\begin{itemize}
  \item SodaHelper.py
  \item producer.py
\end{itemize}

Das SodaHelper Skript ist ein separatzer Wrapper um die sodapy Bibliothek um die Verbindung zu der API herzustellen und die Daten zu holen.
Somit wird eine klare Aufgabentrennung erreicht.
Das Skript SodaHelper ist für die Verbindung zu der API zuständig und das producer Skript nur für die Weiterleitung der empfangenen Daten an Apache Kafka.

Da es sich hierbei nicht um ein Live Stream handelt wie \zb{} bei Twitter API, haben wir uns dazu entschieden einen "Fake Stream" zu erstellen,
indem nicht alle Daten sofort an Apache Kafka weitergeleitet werden sondern immer ein gewisser Abstand zwischen dem Senden der einzelnen Datensätze
erzwungen wird.

Unser ausgewählter Datensatz ist sehr groß. Wenn \zb{} der Nutzer Daten von einem Monat abrufen will kann es vorkommen, dass in diesem Zeitraum mehr als 100.000 Datensätze
bereit zum Abruf stehen.
Da der Abruf von einer solch großen Menge an Daten über die API eine sehr hohe Rechenleistung erfordert und diese im Rahmen des Projektes nicht zur Verfügung steht wurde
eine Art \textit{Paging} entwickelt.
Durch die feste Angabe eines Limits von 10.000 wird sichergestellt, dass ein Request an die \ac{SODA} \ac{API} nur maximal 10.000
Datensätze liefern kann und durch einen entwickelten Algorithmus werden so lange Requests ausgeführt bis der gewünschte Zeitraum des Users komplett empfangen und
die Request an Apache Kafka gesendet wurden.

Vorgegeben durch das Framework \textit{kafka-python} können die Einträge eines Topics nur als byte String abgelegt werden.
Aus diesem Grund wird der empfangende Datensatz zuerst in ein byte string umgewandelt bevor er an Apache Kafka gesendet wird.

Folgendes Code Snippet zeigt den entwickelten Algorithmus um das Paging zu realisieren.
Mit Hilfe des SodaHelpers werden zunächst die Datensätze von der API - unter Berücksichtigung des Limits, des Anfangs- und Enddatums geholt.
Wie in dem Snippet zu erkennen wird die Variable \textit{limit} in dem Skript gesetzt.
die Variablen \textit{from\_date} und \textit{to\_date} werden beim Starten des Skripts von dem User gesetzt.

Der Algorithmus beruht auf der Annahme, dass wenn der empfangene Datensatz genau die Länge des Limits hat es immer noch weitere Datensätze gibt die von der API abgerufen werden müssen.
Wenn also das Limit erreicht wurde wird das Datum des letzten Datensatzes als neues Enddatum festgelegt und der Prozess beginnt von vorne, solange die Anzahl der empfanenen Datensätze nicht mehr dem Limit entsprechen
oder die verwendeten Anfangs- und Enddatumswerte identisch sind.

\lstinputlisting[language=Python, firstline=20, lastline=45]{../python/producer.py}

\subsection{Data Ingestion via CSV Datei}\label{subsec:csv}
Als offline-fähige Alternative zur \ac{SODA} API gibt es einen weiteren Kafka Producer in Java.
Dieser liest die Nachrichten zeilenweise aus einer lokalen \ac{CSV} Datei ein und publiziert die Datensätze als \ac{JSON} einzeln an einen Kafka Topic.
Die \ac{CSV} Datei wurde vorher aus dem NYC OpenData Portal runtergeladen.

Der Programmablauf lässt sich in wenigen Stichpunkten beschreiben:
\begin{enumerate}
	\item Verbindung des \code{KafkaProducer<Long, String>} zum Kafka Cluster konfigurieren.Dazu zählen hauptsächlich Host und Port sowie die Datentypen, um Schlüssel und Wert der Nachrichten zu serialisieren.
	\item Zeilenweises einlesen der \ac{CSV} Datei und dabei jeweils den Datensatz in einen \ac{JSON}-String transformieren, diesen \code{String} als Wert in einen \code{ProducerRecord<Long, String>} setzen und an den bestimmten Kafka Topic senden. Um einen realen Stream zu simulieren wartet der Producer-Thread pro verarbeiteten Datensatz eine zufällige Wartezeit zwischen 0 bis 2 Sekunden.
\end{enumerate}

Listing \ref{listing:javacsvproducer} zeigt einen Überblick über die relevanten Methoden.

%\begin{tcblisting}{width=19cm,listing only,blank,tikz={rotate=90},listing options={basicstyle=\ttfamily}}
\lstinputlisting[language=Java, firstline=37, lastline=68, label=listing:javacsvproducer,
	captionpos=b,
	caption=Auszug aus \code{com.srh.bdba.dataengineering.MyProducer}]{../java/src/main/java/com/srh/bdba/dataengineering/MyProducer.java}
%\end{tcblisting}

Zur Implementierung wurden folgende Bibliotheken über Maven eingebunden:
\begin{itemize}
	\item \code{org.apache.kafka:kafka-clients}
	\item \code{org.apache.commons:commons-csv}
	\item \code{com.fasterxml.jackson.core:jackson-core}
	\item \code{com.fasterxml.jackson.core:jackson-databind}
\end{itemize}

