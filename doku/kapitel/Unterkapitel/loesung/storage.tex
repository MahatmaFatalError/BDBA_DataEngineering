\section{Data Storage}

Wie in \fullref{sec:arch} schon erläutert benutzen wir Apache Kafka um die kompletten Datensätze zwischenzuspeichern.
Ein sog. "Consumer" liest die Daten von Apache Kafka aus und speichert die relevanten Teile der Datensätze in der Datenbank.
Als Datenksystem haben wir in unserem Projekt für PostgreSQL entschieden.

In unserer Datenbank erstellten wir eine Tabelle mit dem Namen \textitbf{service\_request} um die Service Requests von New York zu speichern.

In der offiziellen Dokumentation des Datensatzes werden die einzelnen Attribute
eines Service Requests genau beschrieben und deren technische Bezeichner aufgelistet.
\footnote{https://dev.socrata.com/foundry/data.cityofnewyork.us/fhrw-4uyv}

Wir erstellten unsere Datenbanktabelle bzw. die Attribute der Service Requests Tabelle mit
genau den gleichen Bezeichnern wie die des \ac{SODA} Datensatzes.

Um den Consumer technisch umzusetzen benutzen wir die Programmiersprache Python wie auch weitere Bibliotheken um das Handling mit Apache Kafka
und der Datenbank zu vereinfachen.
Diese sind:

\begin{itemize}
  \item kafka-python
  \item sqlalchemy
  \item psycopg2 (wird von sqlalchemy benötigt)
\end{itemize}

Genau wie bei dem "Producer" abstrahiert das Paket \textitbf{kafka-python} die Verbindung zu unserem Apache Kafka Server und erleichtert den Zugriff auf das Topic.
Durch den Einsatz von \textitbf{sqlalchemy} ist es möglich auf die Datenbank und deren Tabelle(n) in einer objektorientierten Weise zuzugreifen
und die Ausführung von \ac{SQL} wird abstrahiert bzw. durch Klassenmethoden abstrahiert.

\subsubsection{Python Quellcode Programmablauf}
\label{subsub:quellcode_storage}
Neben den verwendeten Bibliotheken besteht der \textitbf{Consumer} aus zwei Python Skripten.

\begin{itemize}
  \item DBHelper.py
  \item consumer.py
\end{itemize}

Während sich das \textitbf{consumer} Skript um das Auslesen eines Kafka Topics kümmert ist der \textitbf{DBHelper}
dafür zuständig die Verbindung zu der Datenbank aufzubauen, Tabellenspalten auszulesen und einen Datensatz in der Tabelle zu speichern.

Nachfolgend das Consumer Skript.

\lstinputlisting[language=Python]{../python/consumer.py}

Sobald das Consumer Skript gestartet wird, wird eine Verbindung mit der Datenbank
aufgebaut und der \code{KafkaConsumer} stellt eine Verbindung mit dem Apache Kafka
Server bzw. dem Topic "ServiceRequest" her.
(Zeile 1 - 9)

Sobald von seitens des Producers ein neuer Datensatz in das Topic "ServiceRequests" geschrieben wird,
wird der Datensatz ausgelesen in ein \ac{JSON} umgewandelt, die benötigten Attribute aus dem \ac{JSON} gelesen
und in ein temporäres Dictionary geschrieben.
Dieses Dictionary wird abschließend mit Hilfe des \code{DBHelper} Skripts in die
Datenbanktabelle geladen.
(Zeile 11 - 18)

Wie eingangs erwähnt bezeichneten wir die Attribute der Datenbanktabelle und des \ac{SODA} Datensatzes gleich
und konnten sie somit als Schlüsselattribute für den jeweils anderen Datensatz benutzen.

Mit diesem kleinen "Kniff" können wir jetzt mit Hilfe der Namen der Tabellenspalten
auf die Keys des \ac{JSON} zugreifen den zugehörigen Wert auslesen und als neuen Wert für das temporäre Dictionary nutzen.
(Zeile 14 - 16)

\subsubsection{Java Quellcode Programmablauf}
Auch für den Consumer gibt es eine alternative Implementierung in Java. Diese findet sich in \code{com.srh.bdba.dataengineering.MyConsumer}.
Der Programmablauf ist ähnlich zu der Python Implementierung, weshalb an dieser Stelle auf Codelistings im Anhang TODO verwiesen wird.
Kurz zusammengefasst wird ein \code{KafkaConsumer<Long, String>} konfiguriert und pro 100 Millisekunden am Kafka-Cluster nachgefragt, ob es neue \code{ConsumerRecord <Long, String>} für das entsprechende Topic gibt.
Falls dem so ist wird die \ac{JSON} Nachricht aus dem \code{ConsumerRecord<Long, String>} ausgepackt und per \code{PreparedStatement} über \code{JDBC} in die PostgreSQL Tabelle eingefügt.
