\chapter{Inbetriebnahme}
\label{chap:betrieb}
In diesem Kapitel werden die einzelnen Schritte beschrieben, welche durchgeführt werden müssen um den Prototyp auf einem Windows Rechner ausführen zu können.

\textbf{Hinweis:}
\newline
Hierbei handelt es sich lediglich um eine High-Level Beschreibung der Inbetriebnahme des Prototypen.
Weitere Details zur Instalation und Nutzung können den Quellen entnommen werden.
\newline

\section{Installation der benötigten Tools \& Frameworks}
\begin{itemize}
  \item Apache Kafka und Zookeeper installieren\autocite{Kafka}
  \begin{itemize}
    \item \textbf{Tipp:} Zur Installation unbedingt die \textit{binary} Version von Apache Kafka herunterladen
  \end{itemize}
  \item Apache Kafka Topic mit dem Namen \textitbf{ServiceRequests} erstellen\autocite{KafkaTopic}
  \item PostgreSQL installieren\autocite{PostgreSQL}
\end{itemize}

Für die Durchführung des Prototyps mit Python müssen zusätzlich noch folgende Schritte durchgeführt werden.
\begin{itemize}
  \item Python und pip installieren\autocite{Python}
  \item benötigte Python Bibliotheken installieren
  \begin{itemize}
    \item \shellcmd{pip install sodapy}
    \item \shellcmd{pip install kafka-python}
    \item \shellcmd{pip install psycopg2}
    \item \shellcmd{pip install sqlalchemy}
    \item \shellcmd{pip install jupyter}
    \item \shellcmd{pip install bokeh}
    \item \shellcmd{pip install ipython-sql}
    \item \shellcmd{pip install gmaps}
  \end{itemize}
  \item \texttt{gmaps} Widget für Jupyter aktivieren
  \begin{itemize}
    \item \shellcmd{jupyter nbextension enable --py --sys-prefix widgetsnbextension}
    \item \shellcmd{jupyter nbextension enable --py --sys-prefix gmaps}
  \end{itemize}
  \item Google Maps API Key erstellen um die Google Maps Visualisierungen nutzen zu können
  \item Applikation Token für die \ac{SODA} \ac{API} beziehen\autocite{AppToken}
\end{itemize}

Für die Durchführung des Prototyps mit Java müssen zusätzlich folgende Schritte durchgeführt werden.
\begin{itemize}
  \item ... TODO ...
\end{itemize}

Abschließend kann unser GitHub Repository via \shellcmd{git clone https://github.com/johannesweber/BDBA_DataEngineering.git}
auf den Rechner kopiert werden.
Falls kein git installiert wurde kann man es auch \hyperref[hier]{http://gitforwindows.org/} installieren oder das Repository als .zip Datei herunterladen.

Alle Daten die für die Ausführung des Java Quellcodes benötigt werden befindet sich in dem Ordner \textit{java}
und alle Python Skripte im Ordner \textit{python}.

In dem Ordner \texttt{db} befinden sich die SQL Skripts um die Datenbank und die Datenbanktabelle aufzubauen.

Zusätzlich muss noch die Konfiguration des Programms angepasst werden.
Für Python befindet sich die Konfigurationsdatei in \texttt{\\python\\config.py} und für Java in XXX.

Die Datei mit dem Namen \textit{BDBA\_DataEngineering.ipynb} ist das Jupyter Notebook, das benötigt wird um den Python Teil von Data Retrieval auszuführen
und den Java Teil kann mmit dem Zeppelin Notebook XXX ausgeführt werden.
Beide Notebooks befinden sich im Root Verzeichnis des Repositories.

\section{Ausführung des Prototyps}
\begin{itemize}
  \item Zookeeper starten
  \item Apache Kafka starten
  \item Programm starten
  \begin{itemize}
    \item für Java: /java/Main.java
    \item für Python: /python/main.py
  \end{itemize}
  \item Notebook starten und die Notebook Datei hochladen.
  \begin{itemize}
    \item für Java: XXX
    \item für Python: BDBA\_DataEngineering.ipynb
  \end{itemize}
\end{itemize}
