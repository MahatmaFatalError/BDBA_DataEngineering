\chapter{Inbetriebnahme}
\label{chap:betrieb}
In diesem Kapitel werden die einzelnen Schritte beschrieben, welche durchgeführt werden müssen um den Prototyp ausführen zu können.

\textbf{Hinweis:}
\newline
Hierbei handelt es sich lediglich um eine grob-granulare Beschreibung der Inbetriebnahme des Prototypen.
Weitere Details zur Installation und Nutzung können den Quellen entnommen werden.

\section{Voraussetzung \& Infrastruktur}
%\subsection{Installation}
Folgende Komponente \underline{müssen}  installiert und lauffähig sein:
\begin{itemize}
  \item Apache Kafka und Apache Zookeeper
  %\item Apache Kafka Topic mit dem Namen \textitbf{ServiceRequests} erstellen\autocite{KafkaTopic}
  \item PostgreSQL
\end{itemize}


Je nachdem welche Technologie man für die Consumer, Producer und Auswertung verwenden möchte ergeben sich folgende zwei \underline{alternative} Technologiestacks:
\setlength{\columnseprule}{1pt}
\def\columnseprulecolor{\color{black}}
\begin{multicols}{2}
	\begin{itemize}
		\item Python 2
		\item pip
		\item Jupyter
	\end{itemize}
\columnbreak
	\begin{itemize}
		\item JDK 1.8
		\item Maven
		\item Apache Zeppelin
	\end{itemize}
\end{multicols}

\section{Infrastruktur Setup \& Konfiguration}
\begin{enumerate}
	\item Projekt aus Git Repository\footnote{\hyperref[hier]{https://github.com/johannesweber/BDBA\_DataEngineering.git}} auschecken
	\item Apache Zookeeper starten und initialisieren
	\item Apache Kafka starten und initialisieren
	\item Kafka Topic anlegen
	\item PostgreSQL Datenbankschema initialisieren\footnote{DDL Skripte um das Datenbankschema in PostgreSQL aufzubauen befinden sich im Ordner /db}
\end{enumerate}

\section{Setup und Starten des Python Prototypen }

\begin{enumerate}
  \item Python Bibliotheken installieren
  \begin{itemize}
    \item \shellcmd{pip install sodapy}
    \item \shellcmd{pip install kafka-python}
    \item \shellcmd{pip install psycopg2}
    \item \shellcmd{pip install sqlalchemy}
    \item \shellcmd{pip install jupyter}
    \item \shellcmd{pip install bokeh}
    \item \shellcmd{pip install ipython-sql}
    \item \shellcmd{pip install gmaps}
  \end{itemize}
  \item \texttt{gmaps} Widget für Jupyter aktivieren
  \begin{itemize}
    \item \shellcmd{jupyter nbextension enable --py --sys-prefix widgetsnbextension}
    \item \shellcmd{jupyter nbextension enable --py --sys-prefix gmaps}
  \end{itemize}
  \item Google Maps \ac{API} Key erstellen um die Google Maps Visualisierungen nutzen zu können\footnote{Eine Anleitung zum Erzeugen eines \ac{API} Keys gibt es \hyperref[hier]{http://jupyter-gmaps.readthedocs.io/en/latest/authentication.html}}
  \item Applikation Token für die \ac{SODA} \ac{API} beziehen\autocite{AppToken}
  \item Python Konfigurationsdatei \texttt{/python/config.py} pflegen
  \item Python Consumer und Producer starten via \texttt{/python/main.py}
  \item Jupyter starten und vorkonfiguriertes Notebook BDBA/DataEngineering.ipynb laden
\end{enumerate}

\section{Setup und Starten des Java Prototypen }

\begin{enumerate}
	\item CSV Datei von \hyperref[https://data.cityofnewyork.us/Social-Services/311-Service-Requests-from-2010-to-Present/erm2-nwe9]{https://data.cityofnewyork.us/Social-Services/311-Service-Requests-from-2010-to-Present/erm2-nwe9} runterladen und lokal speichern (ca. 11GB).
	\item Java Projekt mit Maven \shellcmd{mvn clean install} bauen.
	\item Konfiguration \texttt{/java/src/main/resources/kafka\_config.perroperties} pflegen.
	\item Python Consumer und Producer starten via \texttt{com.srh.bdba.dataengineering.Main}. Beim Aufruf der \texttt{Main()} muss man den Pfad zur CSV Datei angeben.
	\item Apache Zeppelin starten und Datenbankverbindung sowie JDBC Treiber im PSQL Interpreter pflegen\footnote{Vgl. Screenshot im Anhang \ref{fig:zeppelin_conf}}.
	\item Vorkonfiguriertes Notebook BDBA/zeppelin\_servicerequests\_notebook.json laden und Abfragen ausführen.
\end{enumerate}
